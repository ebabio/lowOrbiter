\section{Introduction}
The exploration of the outer Solar System has led to many exciting discoveries and a unique opportunity to develop new technologies. Past and present missions have helped us to understand the composition and behavior of these planetary system, but have also raised new questions. The outcome of the Galileo mission to Jupiter and Cassini to Saturn suggest that the icy moons Europa and Enceladus, one on each system, may have liquid oceans beneath the surface that may be capable of sustaining life.

NASA defined the exploration of Europa as the second priority in space exploration \cite{nasa2011vision} and an orbiter mission was conceived in partnership with ESA, Jupiter Europa Orbiter - Laplace, but it was canceled and due to budgetary constraints. While NASA still pursued the exploration of Europa through a fly-by mission, ESA shifted its approach to JUICE: a fly-by mission that would study Europa and other Jovian moons finishing with an injection into Ganymede \cite{esa2011juice} and thus being the first spacecraft to orbit a planetary satellite other than Earth's Moon.

This work will focus on the analysis of the orbital dynamics for a scenario common to all these missions: the motion of the orbiter about a satellite for low altitude orbits. For this, we will develop a first-order analytical theory to gain insight on the general behavior of the orbit which can be used in the preliminary design of the orbits drawn from Scheeres \cite{scheeres2012orbital} \cite{scheeres2001stability}. The novelty of this scenario is that we will have to consider the effect of tidal force of the planet on the orbit besides the oblateness of the satellite. Unlike the perturbation from distant third bodies, the gravitational perturbation due to the central body cannot be considered constant, and this differential force will give rise to new characteristics in the motion.

We will study high inclination orbits since these are desirable for the scientific purposes of such a mission. We will show that the main effect is the effect on the eccentricity and the argument of periapsis. Depending on the inclination of the orbit, we can obtain secular effects on the eccentricity that will lead to instability and eventually to an impact against the surface. However, this behavior can also be leveraged to circularize the orbit as JUICE intends.

This work is split into the following sections. After the introduction, the definition of the model is presented: Hill's problem will be introduced ans used to define the perturbation function. In the analysis section, we will make use of averaging techniques to obtain the secular effect of the perturbations and derive from it the main characteristics of the motion. In the results we will study the motion for some specific cases in the environment of Europa and we will obtain patterns similar to those used in JUICE, and compare it with more especial perturbations solutions. Finally we will present our conclusions.

