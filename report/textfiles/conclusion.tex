\section{Conclusions}

In this project a first-order analytical theory has been developed for the motion of an orbiter about a planetary satellite. The considered perturbations have been the effect of tidal force due to the planet and the $J_2$ harmonic due to the oblateness of the satellite which are averaged over non-commensurate revolutions about the satellite and the planet. The Lagrange planetary equations are obtained and simplified for the case of a small eccentricity orbiter. 

Under these assumptions two kind of behaviors appear for the eccentricity vector, all other elements stay constant or have been averaged. For low inclination orbits the perturbations lead to bounded oscillations in modulus and argument, for high inclination of greater scientific interest a bifurcation leads to an unstable behavior. This behavior is characterized by an equilibrium for circular orbits and an attractive and unstable manifold. The unstable manifold imposes a restriction on the lifetime of the orbiter since the altitude of periapsis decreases and will eventually cause an impact against the satellite. The stable manifold offers the possibility of leveraging the natural motion of the orbiter to circularize the orbit after an elliptical injection.

The results of the theory have been validated against a especial perturbations propagator than implements the same model. Comparison between the theory and the numerical results for an orbiter in Europa show a great degree of agreement and offer valuable insight for the determination of initial conditions that can be used either to delay the impact against the surface or to leverage the attractive manifold for circularizing the orbit after injection. The effects of the short period perturbations cause the orbiter to eventually excite the unstable modes and thus a control action would necessary to maintain the orbit in the long term.

Finally, the results of the theory are compared with those of a more demanding scenario. Mission profiles obtained from this model are compared with ESA's JUICE mission showing a great degree of agreement. In this setup, the behavior of a highly eccentric orbit insertion is analyzed by comparing the results of the theory against the trajectories obtained from globalizing the manifolds. The theory still proves to be able to offer insight, especially in terms of the convergence rate, but numerical work would become increasingly more necessary for mission design purposes.

Future work should focus on obtaining insight in the short term perturbations. While their effect in the long term behavior seems almost negligible, the transformation between instantaneous and mean orbital elements limits the ability to choose initial conditions from the theory alone.